\documentclass[paper=a4, fontsize=11pt]{scrartcl} 
\usepackage[utf8]{inputenc}
\usepackage{amsmath}
\usepackage{amsfonts}
\usepackage{amssymb}
\author{Kim Thuong Ngo}


\usepackage[T1]{fontenc} 
\usepackage{fourier} 

\usepackage{lipsum} 

\usepackage{listings}
\usepackage{graphicx}
\usepackage{tabularx}

\usepackage{sectsty}
\allsectionsfont{\centering \normalfont\scshape} 

\usepackage{fancyhdr} 
\pagestyle{fancyplain} 
\fancyhead{}
\fancyfoot[L]{} 
\fancyfoot[C]{} 
\fancyfoot[R]{\thepage} 
\renewcommand{\headrulewidth}{0pt} 
\renewcommand{\footrulewidth}{0pt}
\setlength{\headheight}{13.6pt}

\numberwithin{equation}{section} 
\numberwithin{figure}{section} 
\numberwithin{table}{section}

\setlength\parindent{0pt} 

\newcommand{\horrule}[1]{\rule{\linewidth}{#1}} 

\title{	
\normalfont \normalsize 
\textsc{Stochastik} \\ [25pt] 
\horrule{0.5pt} \\[0.4cm] 
\huge Aufgaben \\ 
\horrule{2pt} \\[0.5cm] 
}

\author{Kim Thuong Ngo} 

\date{\normalsize\today} 

%----------------------------------------------------------------------------------------

\begin{document}

\maketitle 

\newpage

\tableofcontents

\newpage

%----------------------------------------------------------------------------------------
\section{Wahrscheinlichkeitstheorie}
%-------------------------------------------------
\subsection{Beweis}
Es sei $\Omega$ eine Menge. Beweisen Sie, dass die Potenzmenge $P( \Omega)$ eine $ \sigma$-Algebra ist. \\

z.z. Potenzmenge $P( \Omega)$ ist eine $ \sigma$-Algebra

nach Definition 1.1:
eine Familie $F \subseteq P( \Omega)$ von Teilmengen des Grundraumes $ \Omega$, heißt $ \sigma$-Algebra \\
es soll gelten: 
\begin{enumerate}
\item leere Menge liegt in F
\item aus $A \in F$ folgt $ \Omega \backslash A \in F$
\item für $n \in \mathbb{N}$ sei $A_{n} \in F$, dann gilt $\cup_{n \in \mathbb{N}} A_{n} \in F$
\end{enumerate}

Beweis:
\begin{enumerate}
\item leere Menge liegt in Potenzmenge \\
$\varnothing \in P( \Omega)$
\item aus $A \in P( \Omega)$ folgt $ \Omega \backslash A \in P( \Omega)$
\item für $n \in \mathbb{N}$ sei $A_{n} \in P( \Omega)$, dann gilt $\cup_{n \in \mathbb{N}} A_{n} \in P( \Omega)$
\end{enumerate}

%-------------------------------------------------
\subsection{Beispiel}
Es sei $ \Omega = [0,2]$. Bestimmen Sie die kleinste $ \sigma$-Algebra, welche die Intervalle $[0,1]$ und $[1,2]$ enthält. \\

$F={\varnothing,\Omega,[0,1],[1,2],[0,1),(1,2],[0,2] \backslash {1},{1}}$ 

\begin{enumerate}
\item $\varnothing \in F$
\item $\Omega \backslash [0,1] = (1,2] \in F$ \\
         $\Omega \backslash [1,2] = [0,1) \in F$ \\
         $\Omega \backslash ([0,2] \backslash {1}) = {1} \in F$ \\
\item $[0,1) \cup (1,2] = [0,2] \backslash {1} \in F$
\end{enumerate}

%-------------------------------------------------
\subsection{zweifacher Wurf eines Würfels}
Beschreiben Sie den zweifachen Wurf eines fairen Würfels: Grundraum $ \Omega$, $ \sigma$-Algebra F und Wahrscheinlichkeitsmaß P. Welche Ergebnisse enthält das Ereignis, dass zumindest einmal die Augenzahl 6 gewürfelt wird. Beschreiben Sie die Augensumme als Zufallsvariable (Abbildungsvorschrift, Zustandsraum). Welche Ergebnisse enthält das Ereignis, dass die Augensumme höchsten 6 ist? \\

$\Omega = {1,...,6}^{2}$ \\
$F = P( \Omega)$ \\
$P(A)= \dfrac{|A|}{36} ; A \subseteq \Omega$ \\
$P(A) = \sum _{w \in A} P({w})= \dfrac{1}{36}$ \\
$X: \Omega \rightarrow z= {2,...,12}$ \\
$z=P(Z)$ \\
$X(w_{1},w_{2})=w_{1}+w_{2} ; w_{1}, w_{2} \in \Omega$

%----------------------------------------------------------------------------------------

\end{document}