\documentclass[paper=a4, fontsize=11pt]{scrartcl} 
\usepackage[utf8]{inputenc}
\usepackage{amsmath}
\usepackage{amsfonts}
\usepackage{amssymb}
\author{Kim Thuong Ngo}


\usepackage[T1]{fontenc} 
\usepackage{fourier} 

\usepackage{lipsum} 

\usepackage{listings}
\usepackage{graphicx}
\usepackage{tabularx}

\usepackage{sectsty}
\allsectionsfont{\centering \normalfont\scshape} 

\usepackage{fancyhdr} 
\pagestyle{fancyplain} 
\fancyhead{}
\fancyfoot[L]{} 
\fancyfoot[C]{} 
\fancyfoot[R]{\thepage} 
\renewcommand{\headrulewidth}{0pt} 
\renewcommand{\footrulewidth}{0pt}
\setlength{\headheight}{13.6pt}

\numberwithin{equation}{section} 
\numberwithin{figure}{section} 
\numberwithin{table}{section}

\setlength\parindent{0pt} 

\newcommand{\horrule}[1]{\rule{\linewidth}{#1}} 

\title{	
\normalfont \normalsize 
\textsc{Stochastik} \\ [25pt] 
\horrule{0.5pt} \\[0.4cm] 
\huge Skript \\ 
\horrule{2pt} \\[0.5cm] 
}

\author{Kim Thuong Ngo} 

\date{\normalsize\today} 

%----------------------------------------------------------------------------------------

\begin{document}

\maketitle 

\newpage

\tableofcontents

\newpage

%----------------------------------------------------------------------------------------

\section{Abkürzungen}

W' : Wahrscheinlichkeiten

%----------------------------------------------------------------------------------------

\section{Grundbegriffe}

\subsection{Wahrscheinlichkeitstheorie}

\underline{Notizen}

$P ({x}) = x \epsilon \Omega $

W' , dass das Ergebnis des zufälligen Experiments gleich x ist, sofern ${x} \epsilon F$, also ein Ereignis ist.

W' maß $P: F \rightarrow [0,1]$ mit 
\begin{itemize}
\item $P( \varnothing ) = 0$
\item $P( \Omega ) = 1$
\item $P(A_{1} \cup A_{2} \cup ... ) = P(A_{1})+P(A_{2})+... $
\end{itemize}
für Folgen $A_{1}, A_{2},... \epsilon F$von paarweisen disjunkten Ereignissen.

Beispiel: zweifacher Münzwurf

$\Omega = {0,1} ^{2} = {0,1} x {0,1} = {(0,0),(0,1),(1,0),(1,1)}$

$F = P( \Omega ) = {\varnothing , {(0,0)}, {(0,1)}, ... , {(0,0),(0,1)}, ... , \Omega }$ $\Rightarrow$ 16 Elemente

z.B. $P({(0,0),(0,1)}) = P({(0,0)})+P({(0,1)}) = q^{2}+ pq = q (q+p) = q (1-p+p) = q $

Anzahl der 1er:

$X: \Omega \rightarrow {0,1,2}$ mit
\begin{itemize}
\item $X(0,0)=0$
\item $X(0,1)= X(1,0)=1$
\item $X(1,1)= 2$
\end{itemize}

ODER: $X(w_{1},w_{2})=w_{1}+w_{2}$

z.B. $P(X^{-1}({1})) = P({(0,1),(1,0)})= 2pq$

%----------------------------------------------------

\subsection{Statistik}

\end{document}