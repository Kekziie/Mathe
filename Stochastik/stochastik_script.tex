\documentclass[paper=a4, fontsize=11pt]{scrartcl} 
\usepackage[utf8]{inputenc}
\usepackage{amsmath}
\usepackage{amsfonts}
\usepackage{amssymb}
\author{Kim Thuong Ngo}


\usepackage[T1]{fontenc} 
\usepackage{fourier} 

\usepackage{lipsum} 

\usepackage{listings}
\usepackage{graphicx}
\usepackage{tabularx}

\usepackage{sectsty}
\allsectionsfont{\centering \normalfont\scshape} 

\usepackage{fancyhdr} 
\pagestyle{fancyplain} 
\fancyhead{}
\fancyfoot[L]{} 
\fancyfoot[C]{} 
\fancyfoot[R]{\thepage} 
\renewcommand{\headrulewidth}{0pt} 
\renewcommand{\footrulewidth}{0pt}
\setlength{\headheight}{13.6pt}

\numberwithin{equation}{section} 
\numberwithin{figure}{section} 
\numberwithin{table}{section}

\setlength\parindent{0pt} 

\newcommand{\horrule}[1]{\rule{\linewidth}{#1}} 

\title{	
\normalfont \normalsize 
\textsc{Stochastik} \\ [25pt] 
\horrule{0.5pt} \\[0.4cm] 
\huge Skript \\ 
\horrule{2pt} \\[0.5cm] 
}

\author{Kim Thuong Ngo} 

\date{\normalsize\today} 

%----------------------------------------------------------------------------------------

\begin{document}

\maketitle 

\newpage

\tableofcontents

\newpage

%----------------------------------------------------------------------------------------
\section{Vorwort}
\subsection{Stochastik}

Beschreibungen von zufälligen Vorgängen, z.B.:

\begin{itemize}
\item Münzwurf, Lotto, Würfel, ...
\item Regentropfen, Katastrophen, radioaktiver Zerfall, ...
\item Warteschlangen
\item Zinsenentwicklung, Aktienkurse, ...
\item Abstimmungen, Wahlergebnisse
\item Wetter
\item Vererbung, Mutation
\end{itemize}

\underline{Experimente}

\begin{itemize}
\item zufällige Vorgänge nennt man Experiment
\item das Ergebnis beschreibt den Ausgang des Experiments
\item ein Ereignis ist eine Menge von Ergebnissen, der wir einer Wahrscheinlichkeit zuordnen können
\end{itemize}

\underline{Aufbau}

\begin{itemize}
\item Modellbildung (stochastisches Modell)
Ziel ist die Beschreibung von Experimenten
\item Wahrscheinlichkeitstheorie
Methoden und Werkzeuge für die Untersuchung von stochastischen Modellen
\item Statistik
Justierung und Überprüfung eines Modells anhand von Daten
\end{itemize}

\subsection{Abkürzungen}
W' : Wahrscheinlichkeiten

%----------------------------------------------------------------------------------------
\newpage
\section{Grundbegriffe}
\subsection{Wahrscheinlichkeitstheorie}
\paragraph{Definition 1.1}
Es sei $ \Omega \neq \varnothing$ eine Menge, der Grundraum. Eine Familie $F \subseteq P( \Omega)$ von Teilmengen heißt $ \sigma$-Algebra, wenn gilt:

\begin{itemize}
\item leere Menge $ \varnothing$ liegt in F
\item aus $A \epsilon F$ folgt $ \Omega \setminus A \epsilon F$
\item für $A_{1}, A_{2}, ... \epsilon F$ gilt $A_{1} \cup A_{2} \cup ... \epsilon F$
\end{itemize}

Eine Abbildung $\mu : F \rightarrow [0, \infty ]$ heißt Maß auf $(\Omega, F)$, wenn gilt:

\begin{itemize}
\item $\mu (\varnothing) = 0$
\item $\sigma$-Additivität: Ist $A_{1}, A_{2},A_{3},... $ eine Folge von paarweisen disjunkten Elementen in F, dann gilt $\mu (A_{1} \cup A_{2} \cup ...) = \mu  (A_{1}) + \mu (A_{2}) + ...$ 
\end{itemize}

Das Tripel $( \Omega, F, \mu)$ heißt Maßraum.

\paragraph{Bemerkung 1.2}
Ist $\Omega$ eine beliebige Menge, dann ist $\zeta : P(\Omega) \rightarrow [0, \infty ]$ mit $\zeta (A) = |A| $ ein Maß, welches Zählmaß heißt.

\paragraph{Bemerkung 1.3}
Die Begriffe Länge, Fläche, Volumen sind Maße auf geeigneten $\sigma$-Algebra von $ \mathbb{R}, \mathbb{R}^{2} , \mathbb{R}^{3}$ definiert. Diese $\sigma$-Algebra enthalten alle Intervalle achsenparallele Rechtecke, achsenparallele Quader. Allgemein ist das d-dimensionale Volumen von geeigneten Teilmengen des $\mathbb{R}^{d}$ ein Maß, welches d-dimensionales Lebesgue-Maß heißt.

\paragraph{Definition 1.4}
Ist P eine Maß auf $(\Omega, F)$ mit $P (\Omega) = 1$, so heißt P ein Wahrscheinlichkeitsmaß und das Tripel $(\Omega, F, P)$ Wahrscheinlichkeitsraum. In diesem Fall heißen die Elemente der $\sigma$-Algebra Ereignisse. \\

Interpretation: Es sei $(\Omega, F, P)$ ein Wahrscheinlichkeitsraum. Dann ist $P (A)$ die Wahrscheinlichkeit, dass das Ergebnis des Experiments im Ereignis $A \epsilon F$ liegt.

\paragraph{Definition 1.5}
Es seien $\Omega , Z$ zwei Mengen. Es sei F eine $\sigma$-Algebra über $\Omega$ und z eine $\sigma$-Algebra über Z. Eine Abbildung $X: \Omega \rightarrow Z$ heißt Zufallsvariable, wenn $X^{-1}(B)= {w \epsilon \Omega: X(w) \epsilon B} \epsilon F \forall B \epsilon z$ gilt. In diesem Fall nennen wir Z den Zustandsraum von X und sagen, dass X von $(\Omega, F)$ nach $(Z,z)$ abbildet, kurz $X:(\Omega, F) \rightarrow (Z,z)$.

\paragraph{Beispiel}
\underline{Münzwurf}
\begin{itemize}
\item $p \epsilon [0,1], q = 1-p$
\item $\Omega={Zahl, Kopf}={0,1}$
\item $F = P (\Omega) = { \varnothing, {0}, {1}, \Omega}$
\item $P (\varnothing) = 0, P ({0}) = q, P({1}) = 0, P(\Omega) = 1$
\end{itemize}

Für $p= \dfrac{1}{2}$ ist das Modell ein fairer Münzwurf.

\underline{zweifacher Münzwurf}
\begin{itemize}
\item $p \epsilon [0,1], q = 1-p$
\item $P({(0,0)}) = q^{2}, P({(0,1)})=P({(1,0)})= pq, P({(1,1)})= p^{2}$
\item für $A \subseteq \Omega : P(A) = \sum _{w \epsilon A} P({w})$
\end{itemize}

Die Größe $X: \Omega \rightarrow {0,1,2}$ mit $X(w)=$ "Anzahl der 1er in w" ist eine Zufallsvariable mit Werten in $Z = {0,1,2} (z = P (Z))$

%--------------------------------------------------
\paragraph{Notizen 16.04.18}
$P ({x}) = x \epsilon \Omega $

W' , dass das Ergebnis des zufälligen Experiments gleich x ist, sofern ${x} \epsilon F$, also ein Ereignis ist.

W' maß $P: F \rightarrow [0,1]$ mit 
\begin{itemize}
\item $P( \varnothing ) = 0$
\item $P( \Omega ) = 1$
\item $P(A_{1} \cup A_{2} \cup ... ) = P(A_{1})+P(A_{2})+... $
\end{itemize}
für Folgen $A_{1}, A_{2},... \epsilon F$von paarweisen disjunkten Ereignissen.

Beispiel: zweifacher Münzwurf

$\Omega = {0,1} ^{2} = {0,1} x {0,1} = {(0,0),(0,1),(1,0),(1,1)}$

$F = P( \Omega ) = {\varnothing , {(0,0)}, {(0,1)}, ... , {(0,0),(0,1)}, ... , \Omega }$ $\Rightarrow$ 16 Elemente

z.B. $P({(0,0),(0,1)}) = P({(0,0)})+P({(0,1)}) = q^{2}+ pq = q (q+p) = q (1-p+p) = q $

Anzahl der 1er:

$X: \Omega \rightarrow {0,1,2}$ mit
\begin{itemize}
\item $X(0,0)=0$
\item $X(0,1)= X(1,0)=1$
\item $X(1,1)= 2$
\end{itemize}

ODER: $X(w_{1},w_{2})=w_{1}+w_{2}$

z.B. $P(X^{-1}({1})) = P({(0,1),(1,0)})= 2pq$

%---------------------------------------------------------------------------
\subsection{Statistik}
\paragraph{Definition 1.6}
Ein statistisches Modell wird durch eine Menge $\Omega$ mit $\sigma$-Algebra F, sowie einer Familie von Wahrscheinlichkeitsmaßen Q auf $( \Omega, F)$ beschrieben. In der parametrischen Statistik wird die Familie Q durch einen Index (Parameter) $\theta$ einer Indexmenge (Parametermenge) $\Theta$ beschrieben: $Q = {P_{\theta} : \theta \epsilon \Theta}$.

\paragraph{Definition 1.7}
Eine Stichprobe ist eine Zufallsvariable X (oder ein zufälliger Vektor $(X_{1}, ..., X_{n})$ oder eine zufällige Folge $X_{1}, X_{2}, ...$) von $( \Omega, F)$ in den Stichprobenraum $(Z,z)$. Ein Element x (oder ein Vektor $(x_{1}, ..., x_{n})$ oder eine Folge $x_{1}, x_{2}, ...$) aus dem Stichprobenraum Z heißt Realisation oder Beobachtung. Ist die Stichprobe ein Vektor, so nennen wir dessen Länge den Stichprobenumfang.

\paragraph{Beispiel}
\underline{Münzwurf}
\begin{itemize}
\item $\Theta = [0,1]$
\item $\Omega = {0,1}, F = P(\Omega)$
\item für $\theta \epsilon \Theta: P_{\theta} (\varnothing) = 0, P_{\theta}({0}) = 1- \theta, P_{\theta}({1}) = \theta, P_{\theta}(\Omega) = 1$
\item Stichprobe: $X: \Omega \rightarrow {0,1}$ mit $X(w) = w$
\end{itemize}

\paragraph{Definition 1.8}
Es sei $Q = {P_{\theta} : \theta \epsilon \Theta}$ ein statistisches Modell und es sei X (oder $X_{1}, ..., X_{n}$) eine Stichprobe im Stichprobenraum $(Z,z)$. Ein statistisches Verfahren ist eine Abbildung vom Stichprobenraum Z (oder von $Z^{n}$) ist eine Antwortmenge A.

\begin{itemize}
\item Ist die Antwortmenge A gleich $\Theta$ (oder $g(\Theta)$, wobei g eine Funktion ist), so sagen wir, das Verfahren ist ein Schätzer
\item Ist die Antwortmenge A gleich ${ja, nein}$ (oder ${1,0}$), so sprechen wir von einem Hypothesentest
\item Ist die Antwortmenge eine Familie von Intervallen, dass sprechen wir von einem Verfahren zur Berechnung von Konfidenzintervallen
\end{itemize}

%----------------------------------------------------------------------------------------

\end{document}